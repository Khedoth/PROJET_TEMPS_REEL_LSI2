\documentclass[a4paper, titlepage]{article}

\usepackage[utf8x]{inputenc}		% accents
\usepackage[margin=1in]{geometry}	% marges
\usepackage[francais]{babel}		% langue
\usepackage{graphicx,subfigure}		% images
\usepackage{verbatim}			% texte préformaté
\usepackage{float}			% utilisation d'objet flottant ?
\usepackage{frame}			% texte encadré

\usepackage{fancyhdr}			% package entếte
\pagestyle{fancy}			% style de page
\usepackage{lastpage}			% lastpage, pour le compteur de nb de pages

\renewcommand\headrulewidth{1pt}	% règles pour l'entête
\fancyhead[L]{Projet Temps-Réel}
\fancyhead[R]{\bsc{Enssat}}

\renewcommand\footrulewidth{1pt}	% règles pour le bas de page
\fancyfoot[C]{\today}
\fancyfoot[L]{\bsc{Biache}-\bsc{Nokaya}}
\fancyfoot[R]{\textbf{Page \thepage/\pageref{LastPage}}}
					% nb de pages

\title{Système de gestion de production d'énergie pour un réseau d'éclairage urbain}
\author{Matthieu \bsc{Biache}, Jean-Michel \bsc{Nokaya}}
\date{\today}



\begin{document}

%page de garde
\makeatletter
	\begin{titlepage}
		\centering
		{\large \textsc{École Nationale Supérieure des Sciences Appliquées et de Technologie}}\\
		\textsc{Logiciel et Systèmes Informatiques - Deuxième année}
		% séparateur entre le haut de la feuille et le centre
		\vfill
			\textbf{Projet Temps-Réel LSI2}\\
		\vspace{0.5cm}
			{\LARGE \textbf{\@title}} \\
		\vspace{1cm}
		{\large Matthieu \bsc{Biache}} \\
		\vspace{0.5cm}
		{\large Jean-Michel \bsc{Nokaya}} \\ 
		\vspace{1cm}

		\@date \\

		% séparation entre le centre et le bas de la feuille
		\vfill
			% encadré en bas à gauche
			\includegraphics[height=0.07\textheight]{enssat.png}
			% espace séparant les coins droit et gauche de la page
			% => création de trois espaces de même largeur dans le vfill
			\hfill
			% encadré en bas à droite
			\begin{tabular}{l}
				\large Chargé de cours :\\[0.2cm]
				\large Benoît \bsc{Vozel} \\
				\large Encadrant de projet :\\[0.2cm]
				\large Nicolas \bsc{Estibals}\\
				\vspace{1cm}
			\end{tabular}
	\end{titlepage}
	\makeatother

	%table des matières
	\tableofcontents

	%nouvelle page
	\newpage

\end{document}
